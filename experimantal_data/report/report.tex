\documentclass[a4paper]{article}
\usepackage[utf8]{inputenc}
\usepackage[russian]{babel}
\usepackage{listings}
\usepackage[a4paper]{geometry}
\usepackage{indentfirst}
\usepackage{graphicx}
\usepackage{caption}

\begin{document}

\title{Сравнение вращаемой и сдвиговой множественных развёрток по количеству вычислений целевой функции в задачах без ограничений}
\author{}
\date{}
\maketitle

\section{Реализация алгоритма с множественными развёртками}
Алгоритм реализован на языке C++ с использованием линейных структур данных для хранения поисковой информации.
Сложность выполнения каждой итерации алгоритма $O(k)$, где $k$ --- номер итерации.

Реализация поддерживает полноценную индексную схему, $\varepsilon$-резервирование и локальную адаптацию (схема Маркина-Стронгина).
Поддержки параллельных вычислений нет.

Данная реализация не использует код системы Globalizer.

\section{Классы тестовых задач и методика проведения экспериментов}

Операционные характеристики метода с различными множественнными развёртками сторились на следующих классах задач:
функции Гришагина ($F_{GR}$), GKLS 2d Simple (gklsS2d), GKLS 2d Hard (gklsH2d), GKLS 3d Simple (gklsS3d).

\subsection{Оепрационные характеристики}

\subsection{Среднее количество вычислений целевой функции}

\section{Предварительные выводы}


\end{document}
