\documentclass[a4paper]{article}
\usepackage[utf8]{inputenc}
\usepackage[russian]{babel}
\usepackage{listings}
\usepackage[a4paper]{geometry}
\usepackage{indentfirst}
\usepackage{graphicx}
\usepackage{caption}

\begin{document}

\title{Сравнение вращаемой и сдвиговой множественных развёрток по количеству вычислений целевой функции в задачах без ограничений}
\author{}
\date{}
\maketitle

\section{Реализация алгоритма с множественными развёртками}
Алгоритм реализован на языке C++ с использованием линейных структур данных для хранения поисковой информации.
Сложность выполнения каждой итерации алгоритма $O(k)$, где $k$ --- номер итерации.

Реализация поддерживает полноценную индексную схему, $\varepsilon$-резервирование и локальную адаптацию (схема Маркина-Стронгина).
Поддержки параллельных вычислений нет.

Данная реализация не использует код системы Globalizer.

\section{Классы тестовых задач и методика проведения экспериментов}

Операционные характеристики метода с различными множественнными развёртками сторились на следующих классах задач:
функции Гришагина ($F_{GR}$), GKLS 2d Simple (gklsS2d), GKLS 2d Hard (gklsH2d), GKLS 3d Simple (gklsS3d).

Для каждого класса задач и каждого типа развёртки были предприняты попытки провести следующие эксперименты:
\begin{enumerate}
  \item решить все задачи при одинаковом для всех развёрток значении $r$ с остановкой по попаданию в окрестность известного оптимума;
  \item решить все задачи при одинаковом для всех развёрток значении $r$ с остановкой по точности;
  \item решить все задачи при минимальном допустимом для каждой конфигурации развёртки в отдельности значении параметра $r$ с остановкой по попаданию в окрестность известного оптимума;
  \item решить все задачи при минимальном допустимом для каждой конфигурации развёртки в отдельности значении параметра $r$ с остановкой по точности;
\end{enumerate}

В последних двух случаях подбор минимального значения $r$ такого, что решаются все задачи класса, осуществлялся с точностью 0.1 для каждого типа
развёртки в отдельности и для каждого значения $L$ (количество развёрток).

В связи с тем, что в используемой реализации АГП используются только линейные структуры данных, не для всех классов указанные 4 типа эспериментов были проведены. Решение некоторых задач из сложных классов требует порядка $10^6$ испытаний и занимает несколько часов на одну задачу. В этос случае подобрать минимальное значение $r$ для каждой развёртки очень затратно.

В таблице указаны эксперименты, которые были проведены. Каждый эксперимент включает в себя решение всех задач класса при $l=1,2,3$ для вращаемой развёртки и
$l=1,2,3,4$ для сдвиговой.


\subsection{Опeрационные характеристики}

\subsection{Среднее количество вычислений целевой функции}

\section{Предварительные выводы}


\end{document}
